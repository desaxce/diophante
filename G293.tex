\documentclass[11pt] {article}

\usepackage[margin=1in] {geometry}
\usepackage[utf8]{inputenc}
\usepackage[T1]{fontenc}
\usepackage[francais]{babel}
\usepackage{amsmath,amsthm,amssymb}
\usepackage{stmaryrd}

\newcommand{\N} {\mathbb{N}}
\newcommand{\Z} {\mathbb{Z}}
\newcommand{\PS} {\mathbb{P}}

\setlength{\parindent}{0cm}

\begin{document}

%\renewcommand{\qedsymbol}{\filledbox}

%\title{}
%\date{}
%\predate{}
%\postdate{}
%\author{}

%\maketitle

\textbf{G293. Un mobilier burenien:}\\
Au milieu de la grande place du Palais Diophantien, Zig a installé neuf colonnes
tronquées de marbre blanc aux rayures blanches et noires qui forment les sommets
d'un ennéagone convexe pas nécessairement régulier et sur lesquels reposent neuf
sculptures en fer forgé représentant les lettres ADEHINOPT.\\
Puce qui se place à l'extérieur du polygone, peut ainsi épeler sous un certain
angle, de gauche à droite, les lettres du mot DIOPHANTE. En continuant de marcher
autour du polygone, il lit, par exemple, HAPNTOEID puis ETDNIOAPH, etc...\\
$Q_{1}$ Déterminer le nombre maximum de mots distincts de neuf lettres que Puce
peut lire quand il fait le tour complet de l'ennéagone.\\
$Q_{2}$ Fort du succès médiatique rencontré avec cet original mobilier urbain,
Zig conçoit la même "œuvre d'art" avec un mot de $k$ lettres distinctes placées
aux sommets d'un $k$-gone convexe qui permet une lecture de $2184$ mots distincts
au maximum quand on en fait le tour comple. Déterminer $k$.\\
Pour les plus courageux: quel est le mot choisi par Zig (3 solutions possibles)?\\

\textbf{Solution:}\\
$Q_{1}$ Prenons le cas d'un $n$-gone convexe avec $n > 3$. Traçons toutes les
droites définies par deux sommets quelconques du polygone.\\

Soit deux sommets $M$ et $N$. Alors $(MN)$ définit deux demi-plans: et selon le
demi-plan dans lequel Puce se trouve, il lira soit $M$ avant $N$, soit l'inverse.\\
Notons qu'en traçant toutes les droites possibles, on coupe le plan en sections
qui, de part leur définition (intersections de demi-plans), sont de même nombre
que le nombre de mots que Puce peut lire (on ne prend que les sections extérieures
au polygone). En effet, être dans une telle section
du plan permet de savoir pour chaque couple de lettre, laquelle sera lue en
premier (car on est dans l'un des deux demi-plans définis par ces deux lettres).
Et si l'on sait comment ordonner les lettres deux à deux, on parviendra à
fabriquer un unique mot final. Et la réciproque est vraie.\\

Comptons donc les sections du plan. Pour cela commençons par compter le nombre
de points d'intersections extérieurs au polygone. Soit un quelconque sommet $M$.
Soit un quelconque autre sommet $N$. Comptons le nombre de droites susceptibles
de croiser $(MN)$ à l'extérieur du polygone. Notons $k$ et $n-k-2$ les nombres
de sommets dans chaque demi-plan défini par $(MN)$. Le nombre de points
d'intersections vaut alors: $\frac{k(k-1)}{2} + \frac{(n-k-2)(n-k-3)}{2}$. Et en
faisant bouger $N$ parmi les autres sommets du polygone, on trouve que le nombre
de points d'intersections où le sommet $M$ est impliqué vaut (par symétrie de $k$
et de $n-k-2$):
\[ 2 \sum\limits_{k=0}^{n-2} \frac{k(k-1)}{2}\]

Et on peut ensuite retirer $M$ du polygone (puisqu'on a calculé sa contribution
aux nombres de points extérieurs) et on re-calcule la même valeur pour un 
$(n-1)$-gone. Finalement on trouve un nombre de points d'intersections extérieurs total:
\[2\sum\limits_{i=0}^{n-2} \sum\limits_{k=0}^i \frac{k(k-1)}{2} = \frac{n(n-1)(n-2)(n-3)}{12}\]\\

Prenons maintenant un sommet $M$ quelconque du polygone. On a alors $n-1$ droites
qui passent par ce sommet. Ces $n-1$ droites définissent $n-1$ sections du plan.
Ainsi, pour chaque sommet, on obtient $n-1$ sections. D'où une contribution
$n(n-1)$ pour les sommets du polygone.\\
En ce qui concerne les points extérieurs, chacun d'eux coupe une section déjà
existante en deux, et apporte une contribution unitaire aux nombres de sections.\\

Par suite, on obtient le nombre total de mots que Puce lira: $\frac{n(n-1)(n^2-5n+18)}{12}$.
Dans le cas $n=9$, cela donne $324$ mots.\\


\underline{Note}: Comme on a un nombre fini de droites, et une infinité d'angles selon lesquels
on peut déplacer un point du polygone, on est certain de pouvoir avoir le
nombre maximal de points d'intersections extérieurs pour une certaine configuration
du polygone (c'est-à-dire qu'on n'autorise pas quatre sommets distincts du polygone
à définir deux droites parallèles).\\

$Q_{2}$ Pour $k = 14$, on obtient $2184$ mots. Et les seuls mots de quatorze lettres
distinctes sont: STYLOGRAPHIQUE, LYMPHOBLASTIQUE et PSYCHODYNAMIQUE.
\end{document}
