\documentclass[11pt] {article}

\usepackage[margin=1in] {geometry}
\usepackage[utf8]{inputenc}
\usepackage[T1]{fontenc}
\usepackage[francais]{babel}
\usepackage{amsmath,amsthm,amssymb}
\usepackage{stmaryrd}

\newcommand{\N} {\mathbb{N}}
\newcommand{\Z} {\mathbb{Z}}
\newcommand{\PS} {\mathbb{P}}

\setlength{\parindent}{0cm}

\begin{document}

%\renewcommand{\qedsymbol}{\filledbox}

%\title{}
%\date{}
%\predate{}
%\postdate{}
%\author{}

%\maketitle

\textbf{A1879. Quatre facteurs premiers:}\\
Trouver sans l'aide d'un quelconque automate le plus petit entier $n > 0$ tel que l'entier $n^{2} - 79n + 1601$ est égal au produit de quatre facteurs premiers pas nécessairement distincts.\\

\textbf{Solution:}\\
Soit un tel $n$ minimal. Notons $p_{1}$, $p_{2}$, $p_{3}$, $p_{4}$ dans $\PS$ tels que: \[n^{2} - 79n + 1601 = p_{1}p_{2}p_{3}p_{4}\]
On a alors: \[n = \frac{79 \pm \sqrt{4p_{1}p_{2}p_{3}p_{4} - 163}}{2}\]
Ainsi, il existe $x$ dans $\N$ tel que: $x^{2} + 163 = 4p_{1}p_{2}p_{3}p_{4}$.\\

Puisque $x$ est impair, en notant $x = 2k+1$, on obtient: $k^{2} + k + 41 = p_{1}p_{2}p_{3}p_{4}$.\\
Pour $k$ dans $\llbracket 0; 39\rrbracket$, $k^{2} + k + 41$ est toujours premier.\\
Et les deux premières valeurs de $k$ qui rendent le trinôme composé sont $40$ et $41$:
\begin{itemize}
\item pour $k = 40$: $k^{2} + k + 41 = 41^{2}$, et $41$ est un candidat pour les $p_{i}$, $i = 1, 2, 3, 4$;
\item pour $k = 41$: $k^{2} + k + 41 = 41.43$, et $43$ est un autre candidat.\\
\end{itemize}

On sait que $p_{1}p_{2}p_{3}p_{4}$ ne peut pas être un carré (car $4p_{1}p_{2}p_{3}p_{4} - x^{2} = 163$, et $163 = 2.81 + 1$ est premier).
Donc le plus petit produit $p_{1}p_{2}p_{3}p_{4}$ ne peut pas être $41^{4}$.\\

Et le prochain produit potentiel est $41^{3}.43$, ce qui donne $k$ égal à $41^{2} + 40 = 1721$ (on cherchait $k$ congru à $40$ modulo $41$), ou encore $x = 3443$ et $n = 1761$.\\

Réciproquement, avec $n = 1761 = 41.43 - 2$, on vérifie l'égalité voulue.

\end{document}
