\documentclass[11pt] {article}

\usepackage[margin=1in] {geometry}
\usepackage[utf8]{inputenc}
\usepackage[T1]{fontenc}
\usepackage[francais]{babel}
\usepackage{amsmath,amsthm,amssymb}
\usepackage{stmaryrd}

\newcommand{\N} {\mathbb{N}}
\newcommand{\Z} {\mathbb{Z}}
\newcommand{\PS} {\mathbb{P}}

\setlength{\parindent}{0cm}

\renewcommand{\arraystretch}{1.5}
\begin{document}

%\renewcommand{\qedsymbol}{\filledbox}

%\title{}
%\date{}
%\predate{}
%\postdate{}
%\author{}

%\maketitle

\textbf{A571. À la recherche du gogol:}\\
$Q_{1}$ Déterminer la plus grande puissance de $10$ obtenue en multipliant des entiers distincts $\leq 2016$. Donner une liste de ces entiers aussi courte que possible.\\
$Q_{2}$ Déterminer le plus petit entier $n_{0}$ tel qu'on sait trouver des entiers $\leq n_{0}$ dont le produit est égal à un gogol $= 10^{100}$.\\

\textbf{Solution:}\\
$Q_{1}$ Comme $10 = 2.5$, les seuls éléments pouvant contribuer à une puissance de $10$ sont de la forme $2^{k}5^{l}$, $\left( k,l \right) \in \N^{2}$.\\

Notons que la plus grande puissance de $5$ inférieure ou égale à $2016$ est $5^{4} = 625$. Et on eut alors construire le tableau des entiers pouvant contribuer à la puissance de $10$ maximale:
\begin{center}
	\begin{tabular}{cccll}
		Total & Contribution $5/2$	& Puissance de $5$	& Puissance de $2$	& Entier suivant \\
		\hline
		$8/1$	&	$8/1$	& $5^{4}$	& $2^{0}$, $2^{1}$	& $5^{4}2^{2} = 2500$ \\
		$23/11$	&	$15/10$	& $5^{3}$	& $2^{0}$, $2^{1}$, $2^{2}$, $2^{3}$, $2^{4}$	& $5^{3}2^{5} = 4000$ \\
		$37/32$	&	$14/21$	& $5^{2}$	& $2^{0}$, $2^{1}$, $2^{2}$, $2^{3}$, $2^{4}$, $2^{5}$, $2^{6}$	& $5^{2}2^{7} = 3200$ \\
		$42/42$	&	$9/36$	& $5^{1}$	& $2^{0}$, $2^{1}$, $2^{2}$, $2^{3}$, $2^{4}$, $2^{5}$, $2^{6}$, $2^{7}$, $2^{8}$	& $5^{1}2^{9} = 2560$ \\
		$N/A$	&	$0/55$	& $5^{0}$	& $2^{0}$, $2^{1}$, $2^{2}$, $2^{3}$, $2^{4}$, $2^{5}$, $2^{6}$, $2^{7}$, $2^{8}$, $2^{9}$, $2^{10}$	& $5^{0}2^{11} = 2048$ \\
	\end{tabular}
\end{center}

Puisqu'on a un déficit de $5$ (comparés aux $2$), on commence par remplir par les plus grandes puissances de $5$. On peut descendre ainsi jusqu'à $5^{1}2^{4}$ inclus, ce qui nous donne une contribution identique pour $5$ et pour $2$: $42$.\\
Par suite, la plus grande puissance de $10$ vaut $10^{42}$ et l'ensemble des nombres de cardinal minimal dont le produit vaut $10^{42}$ est: $625$, $1250$, $125$, $250$, $500$, $1000$, $2000$, $25$, $50$, $100$, $200$, $400$, $800$, $1600$, $5$, $10$, $20$, $40$, $80$.\\

$Q_{2}$ En construisant le même tableau que précédemment avec $n = 31250$, on obtient:

\begin{center}
	\begin{tabular}{cccll}
		Total & Contribution $5/2$	& Puissance de $5$	& Puissance de $2$	& Entier suivant \\
		\hline
		$12/1$	&	$12/1$	& $5^{6}$	& $2^{0}$, $2^{1}$	& $5^{6}2^{2} = 62500$ \\
		$32/7$	&	$20/6$	& $5^{5}$	& $2^{i}$, $i \in \llbracket 0; 3 \rrbracket$	& $5^{5}2^{4} = 50000$ \\
		$56/22$	&	$24/15$	& $5^{4}$	& $2^{i}$, $i \in \llbracket 0; 5 \rrbracket$	& $5^{4}2^{6} = 40000$ \\
		$80/50$	&	$24/28$	& $5^{3}$	& $2^{i}$, $i \in \llbracket 0; 7 \rrbracket$	& $5^{3}2^{8} = 32000$ \\
		$96/86$	&	$22/XX$	& $5^{2}$	& $2^{i}$, $i \in \llbracket 0; 10 \rrbracket$	& $5^{2}2^{11} = 51200$ \\
		$N/A$	&	$XX/XX$	& $5^{1}$	& $2^{i}$, $i \in \llbracket 0; 12 \rrbracket$	& $5^{1}2^{13} = 40960$ \\
	\end{tabular}
\end{center}

À partir de la puissance $5^{2}$, il ne faut pas sélectionner les $2^{i}$ avec $i = 9, 10$ car on ne récupère que deux facteurs $5$ contre neuf ou dix facteurs $2$. 
Il est plus intéressant de choisir $5^{1}2^{2}$, $5^{1}2^{3}$, $5^{1}2^{4}$ pour un coût de neuf facteurs $2$ mais trois facteurs $5$ (un de plus).\\

Ainsi, on obtient une puissance de $10$ maximale égale à $10^{102}$ (en choisissant $5^{1}2^{i}$ pour $i = 0, 1, 2, 3, 4, 6$).
On note que $31250$ est le premier entier pour lequel la fonction $f$ qui à un entier associe le logarithme en base dix de la plus grande puissance de $10$ formée
par un produit d'entiers inférieurs ou égaux à cet entier prend une valeur supérieure à $100$.\\

En effet, si l'on regarde la valeur prise par $f$ en $31249$, on est forcé de retirer l'importante contribution de $5^62^{1} = 31250$. Et on ne peut alors qu'atteindre $10^{97}$. Ainsi $n_0 = 31250$ (il suffit de retirer le facteur $100$ du produit par exemple).

\end{document}
