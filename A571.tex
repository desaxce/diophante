\documentclass[11pt] {article}

\usepackage[margin=1in] {geometry}
\usepackage[utf8]{inputenc}
\usepackage[T1]{fontenc}
\usepackage[francais]{babel}
\usepackage{amsmath,amsthm,amssymb}
\usepackage{stmaryrd}

\newcommand{\N} {\mathbb{N}}
\newcommand{\Z} {\mathbb{Z}}
\newcommand{\PS} {\mathbb{P}}

\setlength{\parindent}{0cm}

\renewcommand{\arraystretch}{1.5}
\begin{document}

%\renewcommand{\qedsymbol}{\filledbox}

%\title{}
%\date{}
%\predate{}
%\postdate{}
%\author{}

%\maketitle

\textbf{A571. À la recherche du gogol:}\\
$Q_{1}$ Déterminer la plus grande puissance de $10$ obtenue en multipliant des entiers distincts $\leq 2016$. Donner une liste de ces entiers aussi courte que possible.\\
$Q_{2}$ Déterminer le plus petit entier $n_{0}$ tel qu'on sait trouver des entiers $\leq n_{0}$ dont le produit est égal à un gogol $= 10^{100}$.\\

\textbf{Solution:}\\
$Q_{1}$ Comme $10 = 2.5$, les seuls éléments pouvant contribuer à une puissance de $10$ sont de la forme $2^{k}5^{l}$, $\left( k,l \right) \in \N^{2}$.\\

Notons que la plus grande puissance de $5$ inférieure ou égale à $2016$ est $5^{4} = 625$. Et on peut alors construire le tableau des entiers pouvant contribuer à la puissance de $10$ maximale:\\

\begin{center}
	\begin{tabular}{cccll}
		Total & Contribution $5/2$	& Puissance de $5$	& Puissance de $2$	& Entier suivant \\
		\hline
		$8/1$	&	$8/1$	& $5^{4}$	& $2^{0}$, $2^{1}$	& $5^{4}2^{2} = 2500$ \\
		$23/11$	&	$15/10$	& $5^{3}$	& $2^{0}$, $2^{1}$, $2^{2}$, $2^{3}$, $2^{4}$	& $5^{3}2^{5} = 4000$ \\
		$37/32$	&	$14/21$	& $5^{2}$	& $2^{0}$, $2^{1}$, $2^{2}$, $2^{3}$, $2^{4}$, $2^{5}$, $2^{6}$	& $5^{2}2^{7} = 3200$ \\
		$42/42$	&	$9/36$	& $5^{1}$	& $2^{0}$, $2^{1}$, $2^{2}$, $2^{3}$, $2^{4}$, $2^{5}$, $2^{6}$, $2^{7}$, $2^{8}$	& $5^{1}2^{9} = 2560$ \\
		$N/A$	&	$0/55$	& $5^{0}$	& $2^{0}$, $2^{1}$, $2^{2}$, $2^{3}$, $2^{4}$, $2^{5}$, $2^{6}$, $2^{7}$, $2^{8}$, $2^{9}$, $2^{10}$	& $5^{0}2^{11} = 2048$ \\
	\end{tabular}
\end{center}

Puisqu'on a un déficit de $5$ (comparés aux $2$), on commence par remplir par les plus grandes puissances de $5$. On peut descendre ainsi jusqu'à $5^{1}2^{4}$ inclus, ce qui nous donne une contribution identique pour $5$ et pour $2$: $42$.
\end{document}
