\documentclass[11pt] {article}

\usepackage[margin=1in] {geometry}
\usepackage[utf8]{inputenc}
\usepackage[T1]{fontenc}
\usepackage[francais]{babel}
\usepackage{amsmath,amsthm,amssymb}
\usepackage{stmaryrd}

\newcommand{\N} {\mathbb{N}}
\newcommand{\Z} {\mathbb{Z}}
\newcommand{\PS} {\mathbb{P}}

\setlength{\parindent}{0cm}

\begin{document}

%\renewcommand{\qedsymbol}{\filledbox}

%\title{}
%\date{}
%\predate{}
%\postdate{}
%\author{}

%\maketitle

\textbf{A365. Les nombres prodigieux:}\\
Un nombre est appelé prodigieux s'il est divisible par le produit de ses chiffres non nuls écrits en base $10$.\\
Par exemple l'entier $2016$ est prodigieux car il est divisible par le produit de ses chiffres non nuls égal à 12.\\
$Q_{1}$ Trouver le plus petit entier prodigieux supérieur à $2016$ qui ne contient aucun chiffre $0$. (*)\\
$Q_{2}$ Trouver quatre entiers consécutifs prodigieux $> 10$. (**)\\
$Q_{3}$ Déterminer la longeur maximale d'une suite d'entiers consécutifs prodigieux. (****). Donner un exemple d'une telle suite.\\

\textbf{Solution:}\\
$Q_{1}$ Le premier nombre supérieur à $2016$ à ne pas contenir le chiffre $0$ dans son écriture en base $10$ est $2111$, qui n'est bien sûr pas divisible par $2$. Vient ensuite $2112$, qui lui est bien divisible par $4$ puisque $12$ l'est. Donc le plus petit entier prodigieux supérieur à $2016$ qui ne contient aucun chiffre $0$ est $2112$.\\

$Q_{2}$ En tâtonnant, on trouve par exemple: $1110$ (divisible par $1$), $1111$ (divisible par $1$), $1112$ (pair), $1113$ (divisible par $3$: somme des chiffres égale à $6$, lui-même multiple de $3$).\\

$Q_{3}$ On note tout d'abord que la séquence de $0$ à $12$ est de longueur treize et est constituée uniquement de nombres prodigieux. Prouvons que cela correspond à la longueur maximale autorisée.\\

L'ensemble des longueurs des suites consécutives d'entiers prodigieux est majoré par $99$ puisqu'aucune suite ne pourrait contenir un nombre congru à $51$ modulo $100$ (car il n'est pas divisible par $5$).
Prenons donc une séquence de taille maximale, $S$ (qui existe bien, et est de taille supérieure ou égale à $13$).\\

\textit{Point 1:} Le critère de divisibilité par $3$ nous permet d'affirmer que $S$ ne contient pas deux entiers se terminant par $3$ en base $10$. En effet, si $x = 3 \mod 10$ (et donc aussi $x+10 = 3 \mod 10$), alors $x+10 = x+1 \mod 3$, et on ne peut pas avoir à la fois $x$ et $x+10$ divisibles par $3$.\\

\textit{Point 2:} De même, avec le critère de divisibilité par $9$, on conclut que $S$ ne peux pas contenir deux nombres congrus à $9$ modulo $10$.\\

Remarquons qu'un nombre congru modulo $100$ à: $14$, $25$, $34$, $41$, $54$, $65$, $74$, $81$, $94$ ou $98$ n'est jamais prodigieux. Et l'espace maximal disponible pour $S$ entre deux de ces nombres est au maximum $12$ (entre $81$ et $94$ exclus par exemple), ce qui ne peut pas correspondre au cardinal de $S$.

Notons également, que si l'on commence une séquence d'entiers prodigieux avec un nombre congru à $99$ modulo $100$ alors on aura une suite de taille maximale $9$ (voir \textit{Point 2}).

Reste donc seulement le cas où les éléments de $S$ sont congrus modulo $100$ à des valeurs comprises entre $0$ inclus et $14$ exclus.

Or on ne peut pas mettre un nombre congru à $13$ modulo $100$ dans $S$, car $S$ ne pourrait pas contenir ce nombre moins $10$ (voir \textit{Point 1}).\\

Donc le cardinal maximal de $S$ est $13$.
Cette valeur est atteinte pour $\llbracket 0; 12 \rrbracket$, mais il en existe d'autres:
$\llbracket 111111111111111111000; 111111111111111111012 \rrbracket$ (car $111111$ est un multiple de $7$, que $1000$ est un multiple de $8$ et que $111111111111111111$ est divisible par $9$!).

\end{document}
